\documentclass[11pt]{article}

\usepackage{amsmath}

\begin{document}

\title{Exercise 6}
\author{Jeremy Nguyen}
\date{\today}
\maketitle

\begin{equation*}
  f=ma
\end{equation*}

\begin{multline*}
  a+b+c+d+e+f+g+h+i+j+k+l+m\\
  +n+o+p+q+r+s+t+u+v+w+x+y+z
\end{multline*}

\begin{equation}
  \begin{split}
    % aligns at equal sign &=
    a+b&=c\\
    e+f&=g
  \end{split}
\end{equation}

\begin{align}
  \label{manyeqs1}
  a+b&c    &    l&=h+n \\
  \label{manyeqs2}
  e+f&g    &    m&=i+o \\
\end{align}


\begin{align*}
  a&=b    &    c&=d  &   e&=f \\
  g&=b    &    h&=d  &  k&=f \\
\end{align*}

\begin{equation}
  a^2=b^5+c^2
\end{equation}

\begin{equation}
  \epsilon+\varepsilon = \theta+\vartheta=\phi+\varphi
\end{equation}

\end{document}